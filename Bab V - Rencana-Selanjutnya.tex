% ==========================================
% BAB V RENCANA SELANJUTNYA
% ==========================================
\chapter{RENCANA SELANJUTNYA}
\label{chap:rencana-selanjutnya}

\section{Rencana Implementasi}

\subsection{Langkah-langkah Implementasi}

Implementasi penelitian prediksi PM\textsubscript{10} Jakarta dilakukan melalui enam tahap utama dengan timeline total 11 minggu. Tahapan-tahapan tersebut dirancang secara sekuensial untuk memastikan kualitas \textit{research} yang optimal dan \textit{manageability} dari segi waktu dan \textit{resources}.

\begin{table}[h]
  \begin{tabular}{ | c | p{3cm} | p{6cm} | c |}
  \hline
  \textbf{No.} & \textbf{Tahap} & \textbf{Kegiatan} & \textbf{Durasi} \\
  \hline
  1 & Persiapan Data & Pengumpulan dataset ISPU Jakarta dari Kaggle, \textit{exploratory data analysis} & 2 minggu \\
  \hline
  2 & Data Preprocessing & Penanganan \textit{missing values}, agregasi multi-stasiun, persiapan data untuk \textit{feature engineering} & 1 minggu \\
  \hline
  3 & Feature Engineering & Pembuatan 15 fitur (5 baseline + 10 \textit{engineered}) meliputi lag, \textit{rolling statistics}, temporal, dan interaksi & 1.5 minggu \\
  \hline
  4 & Pengembangan Model & \textit{Training} dan \textit{hyperparameter tuning} untuk Random Forest, ARIMA, dan model \textit{hybrid} & 3 minggu \\
  \hline
  5 & Integrasi SHAP & Implementasi SHAP untuk \textit{global} dan \textit{local explainability}, visualisasi kontribusi fitur & 1.5 minggu \\
  \hline
  6 & Evaluasi \& Dokumentasi & Pengujian model, analisis hasil, penulisan laporan akhir & 2 minggu \\
  \hline
  & & \textbf{TOTAL} & \textbf{11 minggu} \\
  \hline
  \end{tabular}
\caption{Timeline implementasi penelitian}
\label{tbl:timeline}
\end{table}

Tahap-tahap ini disusun dengan mempertimbangkan ketergantungan antar tahap dan keseimbangan antara \textit{quality} dan \textit{feasibility} dalam konteks jadwal semester di ITB.

\subsection{Alat yang Dibutuhkan}

Penelitian ini menggunakan perangkat keras, perangkat lunak, dan \textit{libraries} yang telah teruji untuk penelitian \textit{machine learning} dalam konteks \textit{time series forecasting}. Berikut adalah daftar alat yang dibutuhkan:

\begin{longtable}{@{\extracolsep{\fill}} c p{2.5cm} p{2.5cm} p{5cm}}
\caption{Alat dan bahan penelitian}\label{tbl:alat} \\
\toprule
\textbf{No.} & \textbf{Kategori} & \textbf{Alat/Bahan} & \textbf{Keterangan} \\
\midrule
\endfirsthead

\caption{Alat dan bahan penelitian (lanjutan)} \\
\toprule
\textbf{No.} & \textbf{Kategori} & \textbf{Alat/Bahan} & \textbf{Keterangan} \\
\midrule
\endhead

\midrule
\multicolumn{4}{r}{\textit{Bersambung ke halaman berikutnya}} \\
\endfoot

\bottomrule
\endlastfoot

1 & Perangkat Keras & MacBook Air M3 2024 & Laptop dengan spesifikasi: CPU Apple M3, RAM 16 GB, Storage 512 GB. Cukup untuk \textit{testing} dan dokumentasi lokal. \\
2 & Bahasa Pemrograman & Python 3.10+ & Bahasa pemrograman untuk implementasi model, dipilih karena ekosistem \textit{library machine learning} yang lengkap. \\
3 & Perangkat Lunak - Data & pandas, NumPy & \textit{Library} untuk manipulasi data dan komputasi numerik. Pandas untuk \textit{structured data handling}, NumPy untuk operasi \textit{array}. \\
4 & Perangkat Lunak - Visualisasi & Matplotlib, Seaborn & \textit{Library} untuk visualisasi data dan hasil analisis. Matplotlib untuk \textit{basic plotting}, Seaborn untuk \textit{statistical visualizations}. \\
5 & Perangkat Lunak - Time Series & statsmodels & \textit{Library} untuk analisis deret waktu dan pemodelan ARIMA, menyediakan API lengkap dan \textit{diagnostic tools}. \\
6 & Perangkat Lunak - ML & scikit-learn, XGBoost & scikit-learn untuk implementasi Random Forest. XGBoost digunakan sebagai alternatif \textit{baseline model}. \\
7 & Perangkat Lunak - XAI & SHAP & \textit{Library} untuk interpretasi model menggunakan Shapley values, menyediakan \textit{global} dan \textit{local explainability}. \\
8 & IDE & Jupyter Notebook & Lingkungan interaktif untuk pengembangan, pengujian, dan dokumentasi. Dijalankan via Google Colab Pro. \\
9 & Version Control & Git, GitHub & Digunakan untuk \textit{version control} dan kolaborasi. Seluruh kode akan disimpan di \textit{repository} GitHub. \\
10 & Cloud Computing & Google Colab Pro & Lingkungan komputasi awan berbayar dengan GPU yang lebih cepat dan \textit{reliable}, 100 unit komputasi per bulan. Digunakan untuk \textit{training} model dengan estimasi waktu 20--30 menit per iterasi. \\
11 & Dataset & ISPU Jakarta 2010--2025 & Dataset dari Kaggle (2010--2025). \\
\end{longtable}

Pemilihan \textit{tools} tersebut didasarkan pada beberapa pertimbangan utama:

\begin{enumerate}

\item Semua \textit{software} adalah \textit{open-source} dan gratis, tidak ada biaya lisensi yang membebani penelitian (kecuali Google Colab Pro untuk \textit{training acceleration}).

\item Ekosistem yang \textit{mature} dan \textit{well-documented}, dengan komunitas pengguna yang besar sehingga memudahkan \textit{troubleshooting} dan \textit{knowledge sharing}.

\item \textit{Support} untuk \textit{Mac architecture} (Apple Silicon M3), memastikan kompatibilitas dengan \textit{hardware} yang digunakan tanpa perlu \textit{dual-boot} atau \textit{virtual machine}.

\item \textit{Best practices} dalam industri untuk \textit{research} serupa, \textit{tools} ini sudah digunakan di berbagai publikasi penelitian kualitas udara dan \textit{time series forecasting}.

\end{enumerate}

Pemanfaatan Google Colab Pro memungkinkan \textit{training} model dengan GPU yang \textit{reliable} dan \textit{efficient}, tanpa membebani \textit{resources} MacBook. Internet yang digunakan sudah termasuk dalam biaya kosan dan akses ITB, sehingga tidak ada biaya tambahan untuk konektivitas. Dengan demikian, seluruh \textit{pipeline research} dapat berjalan dengan performa optimal.

\subsection{Analisis Biaya Implementasi}

Biaya pengembangan model mencakup perangkat keras (MacBook Air M3 sudah dimiliki), perangkat lunak (semua \textit{open-source} gratis), dan Google Colab Pro untuk \textit{accelerated computing}. Internet yang digunakan sudah termasuk dalam biaya kosan dan akses ITB, sehingga tidak ada biaya internet tambahan.

Google Colab Pro dipilih untuk memastikan \textit{training} model yang \textit{reliable} dan \textit{efficient}, dengan GPU yang lebih cepat, 100 unit komputasi per bulan, dan waktu \textit{idle timeout} yang lebih panjang. Dengan \textit{resources} ini, estimasi waktu \textit{training} berkurang menjadi 20--30 menit per iterasi, memastikan timeline 11 minggu tetap \textit{feasible}.

Model dirancang sebagai \textit{open-source} dan dapat diimplementasikan secara gratis di server lokal Dinas Lingkungan Hidup DKI Jakarta, tanpa biaya lisensi atau \textit{subscription}. Dengan pendekatan ini, tidak ada \textit{barrier} finansial untuk adopsi model oleh pemerintah, sejalan dengan tujuan \textit{social impact} dari penelitian ini.

\begin{table}[h]
  \begin{tabular}{ | c | p{3cm} | p{2.5cm} | p{2.5cm} | p{4cm} |}
  \hline
  \textbf{No.} & \textbf{Kategori} & \textbf{Item} & \textbf{Biaya} & \textbf{Catatan} \\
  \hline
  1 & Pengembangan & Hardware (MacBook M3) & Rp 0 & Sudah dimiliki \\
  \hline
  2 & Pengembangan & Software tools (Python, libraries) & Rp 0 & \textit{Open-source}, gratis selamanya \\
  \hline
  3 & Pengembangan & Google Colab Pro & Rp 170K/bulan & \textit{Reliable} GPU, 100 unit komputasi, \textit{training} lebih cepat \\
  \hline
  4 & Pengembangan & Internet & Rp 0 & Sudah termasuk biaya kosan dan akses ITB \\
  \hline
  5 & Implementasi & Biaya \textit{deployment} pengguna & Rp 0 & \textit{Open-source, local installation} \\
  \hline
  & & \textbf{TOTAL} & \textbf{Rp 470K} & \textbf{Untuk 11 minggu \textit{development}} \\
  \hline
  \end{tabular}
\caption{Estimasi biaya implementasi}
\label{tbl:biaya}
\end{table}

\section{Rencana Evaluasi}

\subsection{Metode Pengujian}

Model prediksi PM\textsubscript{10} akan diuji menggunakan metodologi yang \textit{rigorous} untuk memastikan validitas hasil dan generalisasi ke data baru. Metodologi pengujian mencakup beberapa komponen utama.

Pertama, \textbf{\textit{train-test split} 80:20 temporal}. Data dibagi menjadi 80\% untuk \textit{training} (periode 2010--2023, $\sim$4.430 hari) dan 20\% untuk \textit{testing} (periode 2024--2025, $\sim$1.108 hari). Pembagian dilakukan secara temporal (tidak \textit{random}) untuk menghindari \textit{data leakage} dan mencerminkan \textit{real-world scenario} di mana model diprediksi untuk masa depan.

Kedua, \textbf{perbandingan dengan \textit{baseline}}. Model yang diusulkan (\textit{Hybrid} RF-ARIMA + \textit{Feature Engineering}) akan dibandingkan dengan tiga \textit{baseline}:

\begin{enumerate}

\item Baseline 1: ARIMA tunggal (tanpa Random Forest)

\item Baseline 2: Random Forest tunggal (tanpa ARIMA, dengan 5 fitur \textit{baseline} saja)

\item Baseline 3: \textit{Hybrid} RF-ARIMA tanpa \textit{feature engineering}, sesuai \cite{yenkikar2025explainable}

\end{enumerate}

Ketiga, \textbf{metrik evaluasi}. Performa model dievaluasi menggunakan tiga metrik:

\begin{enumerate}

\item \textbf{RMSE} (\textit{Root Mean Square Error}): Mengukur \textit{magnitude} dari \textit{error}, \textit{sensitive} terhadap \textit{outliers}.

\item \textbf{MAE} (\textit{Mean Absolute Error}): Mengukur \textit{average absolute error}, lebih \textit{robust} terhadap \textit{outliers}.

\item \textbf{R²} (\textit{Coefficient of Determination}): Mengukur proporsi varians yang dijelaskan oleh model (target: $\geq$ 0,85).

\end{enumerate}

Keempat, \textbf{analisis residual}. Residual model dianalisis untuk:

\begin{enumerate}

\item Uji normalitas menggunakan uji \textit{Shapiro-Wilk} untuk memverifikasi bahwa residual terdistribusi normal.

\item Uji autokorelasi menggunakan plot ACF/PACF dan uji \textit{Ljung-Box} untuk mendeteksi korelasi temporal yang tersisa dalam residual.

\item Deteksi \textit{overfitting} dengan membandingkan metrik (RMSE, MAE, R²) antara data \textit{training} dan \textit{test} untuk memastikan model tidak hanya menghafal data \textit{training}.

\end{enumerate}

\subsection{Kriteria Keberhasilan}

Kesuksesan penelitian dievaluasi berdasarkan lima kriteria yang terukur dan objektif, yang semuanya terkait dengan kebutuhan fungsional (F1-F6) dan nonfungsional (NF1-NF4) yang telah didefinisikan pada Bab III.

\begin{table}[h]
  \begin{tabular}{ | c | p{3cm} | p{3.5cm} | p{3.5cm} |}
  \hline
  \textbf{No.} & \textbf{Kriteria} & \textbf{Target} & \textbf{Keterangan} \\
  \hline
  1 & RMSE model \textit{hybrid} lebih rendah dari \textit{baseline} terbaik & $\geq$ 15\% \textit{improvement} & Menjawab kebutuhan NF1 (akurasi) \\
  \hline
  2 & R² model \textit{hybrid} & $\geq$ 0,85 & Menjawab kebutuhan F1 dan F4 (prediksi akurat) \\
  \hline
  3 & Model tidak \textit{overfitting} & Selisih R² \textit{train-test} $<$ 0,10 & Menjawab kebutuhan NF2 (\textit{robustness}) \\
  \hline
  4 & SHAP mengidentifikasi \textit{top} 5 fitur penting & Ya, teridentifikasi dengan jelas & Menjawab kebutuhan F5 dan NF3 (\textit{explainability}) \\
  \hline
  5 & Eksperimen dapat direplikasi & Skrip Python terdokumentasi di GitHub & Menjawab kebutuhan NF4 (\textit{reproducibility}) \\
  \hline
  \end{tabular}
\caption{Kriteria keberhasilan penelitian}
\label{tbl:kriteria}
\end{table}

Kelima kriteria ini dirancang agar \textit{objective}, \textit{measurable}, dan \textit{achievable} dalam timeline 11 minggu dengan \textit{resources} yang tersedia. Jika semua kriteria terpenuhi, penelitian dianggap berhasil mengembangkan model prediksi PM\textsubscript{10} Jakarta yang akurat, \textit{robust}, dan \textit{interpretable}.

\section{Analisis Risiko}

Bagian ini mengidentifikasi risiko potensial yang mungkin menghambat penelitian beserta strategi mitigasi yang realistis dan \textit{actionable}.

\begin{longtable}{@{\extracolsep{\fill}} c p{1.8cm} p{2.2cm} p{2cm} p{3.6cm}}
\caption{Analisis risiko dan strategi mitigasi}\label{tbl:risiko} \\
\toprule
\textbf{No.} & \textbf{Risiko} & \textbf{Penyebab} & \textbf{Dampak} & \textbf{Mitigasi} \\
\midrule
\endfirsthead

\caption{Analisis risiko dan strategi mitigasi (lanjutan)} \\
\toprule
\textbf{No.} & \textbf{Risiko} & \textbf{Penyebab} & \textbf{Dampak} & \textbf{Mitigasi} \\
\midrule
\endhead

\midrule
\multicolumn{5}{r}{\textit{Bersambung ke halaman berikutnya}} \\
\endfoot

\bottomrule
\endlastfoot

1 & Keterbatasan \textit{hardware} untuk \textit{training} & Proses \textit{training} model kompleks memerlukan komputasi besar; MacBook M3 hanya mengandalkan CPU & \textit{Training} berlangsung lama, berpotensi melewati jadwal yang direncanakan & Menggunakan Google Colab Pro dengan dukungan GPU yang \textit{reliable}. Colab Pro menyediakan \textit{resources} yang cukup untuk model ini dengan estimasi \textit{training} 20--30 menit per iterasi. \\
2 & \textit{Data quality issues} (\textit{missing values} terlalu banyak) & Data ISPU dari berbagai sumber mungkin tidak lengkap atau tidak konsisten & Akurasi model menurun dan hasil prediksi menjadi kurang dapat dipercaya & Menggunakan beberapa metode imputasi (\textit{rolling mean}, interpolasi). Jika persentase data hilang $>$10\%, menambahkan data dari BMKG atau stasiun cuaca alternatif sebagai pendukung. \\
3 & Model tidak mencapai target akurasi R² $\geq$ 0{,}85 & \textit{Feature selection} kurang optimal atau \textit{hyperparameter tuning} tidak konvergen & Kriteria keberhasilan tidak terpenuhi, sehingga perlu revisi desain atau timeline & Melakukan \textit{intensive hyperparameter tuning} menggunakan \textit{Grid Search}. Mencoba kombinasi \textit{hyperparameter} yang berbeda (misalnya jumlah pohon RF, \textit{order} ARIMA p-d-q). Mencoba metode \textit{ensemble} lain seperti \textit{Stacking} atau \textit{Voting} menggunakan kombinasi RF, ARIMA, dan XGBoost. Jika tetap tidak tercapai, dokumentasikan sebagai \textit{limitation} dan investigasi \textit{root cause} di hasil penelitian. \\
4 & Komunikasi dengan pembimbing terhambat & Jadwal pembimbing padat, waktu respon lambat, atau terjadi mis-komunikasi terkait arah penelitian & \textit{Progress} penelitian melambat, muncul banyak \textit{rework}, dan \textit{deadline} berpotensi terlewat & Menjadwalkan pertemuan rutin (misalnya dua minggu sekali). Mengirimkan \textit{progress report} tertulis melalui GitHub atau email untuk memungkinkan umpan balik asinkron. Menjaga dokumentasi yang rapi agar diskusi lebih efektif. \\
5 & Timeline tidak realistis & Adanya \textit{unexpected issues} (\textit{bug}, masalah data), atau \textit{underestimation effort} & \textit{Deadline} terlewat dan kualitas hasil penelitian menurun & Menyediakan \textit{buffer} waktu sekitar 20\% di setiap tahap. Memprioritaskan \textit{core deliverables}, sedangkan fitur tambahan dapat ditunda. Menerapkan pendekatan \textit{Agile} dengan \textit{weekly sprint review} untuk memantau progres dan menyesuaikan rencana. \\
\end{longtable}

Semua risiko yang teridentifikasi memiliki strategi mitigasi yang \textit{concrete} dan \textit{feasible} dengan \textit{resources} yang ada. Strategi tersebut dirancang agar \textit{proactive} (mencegah risiko terjadi) dan \textit{reactive} (\textit{handle} jika risiko terjadi). \textit{Monitoring} berkelanjutan terhadap risiko akan dilakukan setiap minggu untuk memastikan \textit{early warning} dan \textit{quick corrective actions} jika diperlukan.