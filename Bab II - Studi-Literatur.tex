% ==========================================
% BAB II STUDI LITERATUR
% ==========================================
\chapter{Studi Literatur}

\section{Kualitas Udara dan PM\textsubscript{10}}

Kualitas udara merupakan indikator penting kesehatan lingkungan yang dipengaruhi oleh berbagai polutan. Salah satu polutan utama adalah \textit{Particulate Matter} (PM), yaitu partikel padat atau cair yang tersuspensi di udara. PM dikategorikan berdasarkan diameter aerodinamisnya, dengan PM\textsubscript{10} merujuk pada partikel dengan diameter kurang dari 10 mikrometer dan PM\textsubscript{2,5} merujuk pada partikel dengan diameter kurang dari 2,5 mikrometer \cite{carb2025inhalable}.

PM\textsubscript{10} dan PM\textsubscript{2,5} memiliki karakteristik dan dampak kesehatan yang berbeda seperti ditunjukkan pada Tabel \ref{tbl:pm10vspm25}. PM\textsubscript{10} umumnya berasal dari sumber mekanis seperti debu jalan, aktivitas konstruksi, dan keausan ban kendaraan, sedangkan PM\textsubscript{2,5} lebih banyak berasal dari proses pembakaran seperti emisi kendaraan dan industri \cite{firdaus2023jakarta}.

\begin{table}[h]
  \begin{tabular}{ | p{3cm} | p{5cm} | p{5cm} |}
  \hline
  \textbf{Aspek} & \textbf{PM\textsubscript{10}} & \textbf{PM\textsubscript{2,5}} \\
  \hline
  Ukuran & $<$ 10 mikrometer & $<$ 2,5 mikrometer \\
  \hline
  Sumber utama & Debu jalan, konstruksi, keausan ban & Pembakaran, emisi kendaraan, industri \\
  \hline
  Penetrasi & Saluran pernapasan atas & Saluran pernapasan dalam, alveoli \\
  \hline
  Dampak kesehatan & Iritasi, asma, COPD & Kardiovaskular, stroke, kanker paru \\
  \hline
  \end{tabular}
\caption{Perbandingan karakteristik PM\textsubscript{10} dan PM\textsubscript{2,5}}
\label{tbl:pm10vspm25}
\end{table}

Dampak kesehatan PM\textsubscript{10} telah terdokumentasi dengan baik dalam literatur. \textcite{zhang2021association} melaporkan bahwa setiap peningkatan 10 µg/m³ konsentrasi PM\textsubscript{10} berhubungan dengan peningkatan kunjungan unit gawat darurat sebesar 0,14\% untuk penyakit kardiovaskular dan 0,56\% untuk aritmia. Pada sistem pernapasan, \textcite{kyung2020particulate} menemukan bahwa peningkatan 10 µg/m³ PM\textsubscript{10} berasosiasi dengan peningkatan 2,7\% pada hospitalisasi \textit{Chronic Obstructive Pulmonary Disease} (COPD) dan 1,1\% pada mortalitas COPD.

\begin{figure}[h]
  \centering
  \captionsetup{justification=centering}
      \includegraphics[width=0.9\textwidth]{image/gambar6.png}
  \caption{Dataset ISPU Jakarta 2010--2025 dari Kaggle menunjukkan distribusi 6 polutan utama}
  \label{gambar:dataset-ispu}
\end{figure}

Dataset ISPU Jakarta yang digunakan dalam penelitian ini mencakup periode 2010--2025 yang terdiri dari data temporal (tanggal, stasiun), konsentrasi lima polutan utama (PM\textsubscript{10}, SO\textsubscript{2}, CO, O\textsubscript{3}, NO\textsubscript{2}), dan nilai ISPU agregat dari lima stasiun pemantauan (DKI1--DKI5). Gambar \ref{gambar:dataset-ispu} menunjukkan distribusi konsentrasi keenam polutan utama dengan PM\textsubscript{10} menunjukkan variasi yang signifikan (rentang 2--187 µg/m³), mengindikasikan pentingnya prediksi akurat untuk sistem peringatan dini.

Di Jakarta, PM\textsubscript{10} merupakan polutan dominan yang berasal dari kombinasi lalu lintas padat dengan lebih dari 20 juta kendaraan bermotor, aktivitas konstruksi yang intensif, dan debu jalan akibat kondisi infrastruktur yang bervariasi \cite{firdaus2023jakarta}. Kondisi ini menjadikan prediksi PM\textsubscript{10} sangat relevan untuk sistem peringatan dini kualitas udara Jakarta.

\section{Indeks Standar Pencemar Udara (ISPU)}

Indeks Standar Pencemar Udara (ISPU) adalah angka tanpa satuan yang menggambarkan kondisi mutu udara ambien di lokasi dan waktu tertentu berdasarkan dampak terhadap kesehatan manusia, nilai estetika, dan makhluk hidup lainnya \cite{permenlhk2020ispu}. ISPU diatur dalam Peraturan Menteri Lingkungan Hidup dan Kehutanan Nomor P.14/MENLHK/\allowbreak SETJEN/\allowbreak KUM.1/\allowbreak 7/\allowbreak 2020 yang mendefinisikan tujuh parameter polutan: PM\textsubscript{10}, PM\textsubscript{2,5}, SO\textsubscript{2}, CO, O\textsubscript{3}, NO\textsubscript{2}, dan HC.

ISPU berfungsi sebagai bahan informasi kepada masyarakat tentang kualitas udara di lokasi dan waktu tertentu, bahan pertimbangan pemerintah dalam melakukan upaya pengendalian pencemaran udara, dan sebagai sistem peringatan dini (\textit{early warning system}) bagi masyarakat \cite{chaniago2020ispu}. Kategori ISPU dan dampaknya terhadap kesehatan ditunjukkan pada Tabel \ref{tbl:ispu}.

\begin{table}[h]
  \begin{tabular}{ | c | l | p{7cm} |}
  \hline
  \textbf{Rentang} & \textbf{Kategori} & \textbf{Dampak Kesehatan} \\
  \hline
  0--50 & Baik & Tidak memberikan efek bagi kesehatan \\
  \hline
  51--100 & Sedang & Tidak berpengaruh pada kesehatan manusia atau hewan, namun berpengaruh pada tumbuhan sensitif \\
  \hline
  101--200 & Tidak Sehat & Merugikan manusia atau kelompok yang sensitif \\
  \hline
  201--300 & Sangat Tidak Sehat & Dapat merugikan kesehatan pada sejumlah segmen populasi yang terpapar \\
  \hline
  $>$300 & Berbahaya & Dapat menimbulkan efek kesehatan serius pada seluruh populasi \\
  \hline
  \end{tabular}
\caption{Kategori ISPU dan dampak kesehatan \cite{septiani2024combating}}
\label{tbl:ispu}
\end{table}

\section{\textit{Machine Learning} untuk Prediksi Kualitas Udara}

Pendekatan \textit{machine learning} telah banyak diterapkan untuk prediksi kualitas udara karena kemampuannya menangkap pola kompleks dan non-linier dalam data lingkungan. Penelitian ini menggunakan model \textit{hybrid} yang menggabungkan \textit{Random Forest} dan ARIMA untuk memanfaatkan kelebihan kedua pendekatan tersebut.

\subsection{\textit{Random Forest}}

\textit{Random Forest} adalah algoritma \textit{ensemble learning} yang mengkombinasikan prediksi dari banyak pohon keputusan (\textit{decision trees}) untuk menghasilkan prediksi yang lebih akurat dan robust \cite{alarsy2025pm25}. Setiap pohon dalam \textit{Random Forest} dilatih menggunakan subset acak dari data pelatihan (\textit{bootstrap sampling}) dan subset acak dari fitur pada setiap \textit{split node}.

Untuk tugas regresi, prediksi \textit{Random Forest} dihitung sebagai rata-rata prediksi dari semua pohon dalam ensemble seperti pada Persamaan \ref{eq:rf}.

\begin{equation}
\hat{y}_{RF} = \frac{1}{B} \sum_{b=1}^{B} T_b(x)
\label{eq:rf}
\end{equation}

Keterangan:
\begin{itemize}
\item $\hat{y}_{RF}$ adalah prediksi \textit{Random Forest}
\item $B$ adalah jumlah pohon dalam \textit{forest}
\item $T_b(x)$ adalah prediksi dari pohon ke-$b$ untuk input $x$
\end{itemize}

Keunggulan \textit{Random Forest} untuk prediksi kualitas udara meliputi kemampuan menangani hubungan non-linier antar variabel, robust terhadap \textit{outlier} dan \textit{noise}, dapat menangani data dengan banyak fitur tanpa memerlukan seleksi fitur eksplisit, serta memberikan estimasi \textit{feature importance} yang berguna untuk interpretasi \cite{abdallah2025predicting}.

\subsection{ARIMA (\textit{AutoRegressive Integrated Moving Average})}

ARIMA adalah model statistik klasik untuk analisis dan peramalan data \textit{time series}. Model ARIMA cocok untuk data yang menunjukkan pola temporal seperti \textit{trend} dan musiman. ARIMA terdiri dari tiga komponen utama: \textit{AutoRegressive} (AR), \textit{Integrated} (I), dan \textit{Moving Average} (MA) \cite{yunis2024hybridization}.

Model ARIMA$(p,d,q)$ didefinisikan pada Persamaan \ref{eq:arima}.

\begin{equation}
\phi(L)(1-L)^d y_t = \theta(L)\epsilon_t
\label{eq:arima}
\end{equation}

Keterangan:
\begin{itemize}
\item $y_t$ adalah nilai \textit{time series} pada waktu $t$
\item $L$ adalah \textit{lag operator} ($Ly_t = y_{t-1}$)
\item $\phi(L) = 1 - \phi_1 L - \phi_2 L^2 - \cdots - \phi_p L^p$ adalah polinomial AR
\item $\theta(L) = 1 + \theta_1 L + \theta_2 L^2 + \cdots + \theta_q L^q$ adalah polinomial MA
\item $d$ adalah orde diferensiasi untuk mencapai stasioneritas
\item $\epsilon_t$ adalah \textit{white noise error}
\end{itemize}

Sebelum menerapkan ARIMA, data harus diuji stasioneritas menggunakan uji \textit{Augmented Dickey-Fuller} (ADF). Jika data tidak stasioner, diferensiasi dilakukan hingga data menjadi stasioner \cite{vasconcelos2025stationarity}.

\subsection{Model \textit{Hybrid Random Forest}-ARIMA}

Model \textit{hybrid} menggabungkan \textit{Random Forest} dan ARIMA untuk memanfaatkan kelebihan kedua pendekatan. \textit{Random Forest} efektif menangkap hubungan non-linier antar fitur, sedangkan ARIMA efektif memodelkan pola temporal pada residual \cite{yenkikar2025explainable}. Arsitektur model \textit{hybrid} ditunjukkan pada Gambar \ref{gambar:hybrid}.

\begin{figure}[h]
  \centering
  \captionsetup{justification=centering}
      \includegraphics[width=0.55\textwidth]{image/gambar2.png}
  \caption{Arsitektur model \textit{hybrid Random Forest}-ARIMA \cite{yenkikar2025explainable}}
  \label{gambar:hybrid}
\end{figure}

Proses prediksi \textit{hybrid} terdiri dari empat tahap. Pertama, \textit{Random Forest} dilatih menggunakan fitur polutan dan fitur hasil \textit{engineering} untuk memprediksi nilai PM\textsubscript{10}. Kedua, residual dihitung sebagai selisih antara nilai aktual dan prediksi \textit{Random Forest}. Ketiga, model ARIMA diterapkan pada residual untuk menangkap pola temporal yang tidak tertangkap \textit{Random Forest}. Keempat, prediksi akhir dihitung sebagai penjumlahan prediksi \textit{Random Forest} dan prediksi residual ARIMA seperti pada Persamaan \ref{eq:hybrid}.

\begin{equation}
\hat{y}_{hybrid} = \hat{y}_{RF} + \hat{r}_{ARIMA}
\label{eq:hybrid}
\end{equation}

Keterangan:
\begin{itemize}
\item $\hat{y}_{hybrid}$ adalah prediksi akhir model \textit{hybrid}
\item $\hat{y}_{RF}$ adalah prediksi dari \textit{Random Forest}
\item $\hat{r}_{ARIMA}$ adalah prediksi residual dari ARIMA
\end{itemize}

\textcite{yenkikar2025explainable} melaporkan bahwa model \textit{hybrid} RF-ARIMA mencapai R² = 0,94 untuk prediksi AQI di India, menunjukkan efektivitas pendekatan ini dalam prediksi kualitas udara.

\section{\textit{Feature Engineering} untuk \textit{Time Series}}

\textit{Feature engineering} adalah proses membuat fitur baru dari data yang ada untuk meningkatkan performa model \textit{machine learning}. Dalam konteks prediksi kualitas udara, \textit{feature engineering} dapat mengekstraksi informasi temporal, statistikal, dan interaksi yang tidak tersedia dalam data mentah \cite{naz2024two}.

\textcite{naz2024two} menunjukkan bahwa pendekatan \textit{feature engineering} dua tahap dapat meningkatkan performa prediksi polutan udara sebesar 5--86\% tergantung pada jenis polutan. \textcite{jimenez2024explainable} juga melaporkan bahwa teknik seleksi fitur temporal secara signifikan meningkatkan akurasi prediksi untuk pola \textit{time series} yang kompleks.

Penelitian ini menggunakan 15 fitur yang terdiri dari 5 fitur baseline dan 10 fitur hasil \textit{engineering} seperti ditunjukkan pada Tabel \ref{tbl:features}. Fitur lag menangkap persistensi polusi dari hari-hari sebelumnya. Fitur \textit{rolling statistics} menangkap tren jangka pendek dan volatilitas. Fitur temporal menangkap pola musiman Jakarta (musim kemarau April--Oktober cenderung memiliki polusi lebih tinggi). Fitur interaksi CO $\times$ O\textsubscript{3} menangkap aktivitas fotokimia yang berkontribusi pada pembentukan PM\textsubscript{10} sekunder.

\begin{table}[h]
  \begin{tabular}{ | l | l | p{7cm} |}
  \hline
  \textbf{Kategori} & \textbf{Fitur} & \textbf{Deskripsi} \\
  \hline
  Baseline & pm10 & Konsentrasi PM\textsubscript{10} hari ini \\
  Baseline & so2 & Konsentrasi SO\textsubscript{2} \\
  Baseline & co & Konsentrasi CO \\
  Baseline & o3 & Konsentrasi O\textsubscript{3} \\
  Baseline & no2 & Konsentrasi NO\textsubscript{2} \\
  \hline
  Lag & pm10\_lag1 & PM\textsubscript{10} 1 hari sebelumnya \\
  Lag & pm10\_lag2 & PM\textsubscript{10} 2 hari sebelumnya \\
  Lag & pm10\_lag7 & PM\textsubscript{10} 7 hari sebelumnya \\
  \hline
  \textit{Rolling} & pm10\_ma3 & \textit{Moving average} 3 hari \\
  \textit{Rolling} & pm10\_ma7 & \textit{Moving average} 7 hari \\
  \textit{Rolling} & pm10\_std7 & Standar deviasi 7 hari \\
  \hline
  Temporal & month & Bulan (1--12) \\
  Temporal & is\_weekend & Akhir pekan (0/1) \\
  Temporal & season & Musim (Hujan/Kemarau) \\
  \hline
  Interaksi & co\_times\_o3 & Interaksi CO $\times$ O\textsubscript{3} \\
  \hline
  \end{tabular}
\caption{Daftar 15 fitur yang digunakan dalam penelitian}
\label{tbl:features}
\end{table}

\section{\textit{Explainable AI} dengan SHAP}

\textit{Explainable AI} (XAI) adalah pendekatan untuk membuat model \textit{machine learning} lebih transparan dan dapat diinterpretasi. SHAP (\textit{SHapley Additive exPlanations}) adalah metode XAI yang didasarkan pada teori permainan kooperatif untuk menjelaskan prediksi individual \cite{molnar2025shap}.

SHAP menghitung kontribusi setiap fitur terhadap prediksi menggunakan nilai Shapley seperti pada Persamaan \ref{eq:shap}.

\begin{equation}
\phi_i = \sum_{S \subseteq N \setminus \{i\}} \frac{|S|!(|N|-|S|-1)!}{|N|!} [f(S \cup \{i\}) - f(S)]
\label{eq:shap}
\end{equation}

Keterangan:
\begin{itemize}
\item $\phi_i$ adalah nilai SHAP untuk fitur $i$
\item $N$ adalah himpunan semua fitur
\item $S$ adalah subset fitur yang tidak mengandung $i$
\item $f(S)$ adalah prediksi model menggunakan fitur dalam $S$
\end{itemize}

Nilai SHAP positif menunjukkan fitur meningkatkan prediksi, sedangkan nilai negatif menunjukkan fitur menurunkan prediksi. Jumlah nilai SHAP semua fitur sama dengan selisih antara prediksi model dan nilai baseline \cite{awan2023introduction}.

Dalam konteks prediksi kualitas udara, SHAP memungkinkan identifikasi faktor-faktor utama yang mendorong polusi. \textcite{radjabaycolle2025improving} menunjukkan bahwa integrasi SHAP dengan model prediksi kualitas udara Jakarta memungkinkan identifikasi faktor lingkungan dan antropogenik kunci, mendukung intervensi kebijakan berbasis bukti. \textcite{antonini2024machine} juga melaporkan bahwa SHAP efektif untuk interpretasi model \textit{machine learning} dalam tugas klasifikasi dan regresi kompleks.

\section{Penelitian Terdahulu}

Beberapa penelitian terdahulu telah mengembangkan model prediksi kualitas udara menggunakan berbagai pendekatan \textit{machine learning}. Ringkasan penelitian terdahulu yang relevan ditunjukkan pada Tabel \ref{tbl:related}.

\begin{table}[h]
  \begin{tabular}{ | p{2.5cm} | p{2cm} | p{2.5cm} | p{2cm} | p{3cm} |}
  \hline
  \textbf{Peneliti} & \textbf{Lokasi} & \textbf{Metode} & \textbf{Hasil} & \textbf{Keterbatasan} \\
  \hline
  Yenkikar dkk. (2025) & India & \textit{Hybrid} RF-ARIMA & R² = 0,94 & Hanya 6 fitur mentah, tanpa \textit{feature engineering} \\
  \hline
  Naz dkk. (2024) & Belfast & LSTM + \textit{Feature Engineering} & 5--86\% \textit{improvement} & Tidak menggunakan model \textit{hybrid} \\
  \hline
  Chen dkk. (2025) & Beijing & KSC-ConvLSTM & RMSE $\downarrow$ 4--8 & Memerlukan data spasial kompleks \\
  \hline
  Al Arsy dan Yasir (2025) & Jakarta & \textit{Random Forest} & R² = 0,61 & Hanya PM\textsubscript{2,5}, tanpa \textit{hybrid} \\
  \hline
  Radjabaycolle dkk. (2025) & Jakarta & LSTM + SHAP & -- & Tidak menggunakan \textit{hybrid} RF-ARIMA \\
  \hline
  \end{tabular}
\caption{Ringkasan penelitian terdahulu}
\label{tbl:related}
\end{table}

\textcite{yenkikar2025explainable} mengembangkan model \textit{hybrid} RF-ARIMA dengan SHAP untuk prediksi AQI di India yang mencapai R² = 0,94. Namun, penelitian tersebut secara eksplisit menyatakan keterbatasan bahwa model mengecualikan faktor eksternal dan hanya menggunakan 6 fitur polutan mentah tanpa mengeksplorasi \textit{feature engineering}.

\textcite{naz2024two} menunjukkan pentingnya \textit{feature engineering} dengan pendekatan dua tahap yang menghasilkan 22 fitur dan meningkatkan performa model LSTM sebesar 5--86\%. \textcite{chen2025hybrid} mengembangkan model KSC-ConvLSTM untuk prediksi PM\textsubscript{2,5} di Beijing yang memanfaatkan data spasial dan temporal, namun memerlukan infrastruktur data yang kompleks.

Untuk konteks Jakarta, \textcite{alarsy2025pm25} mengembangkan model \textit{Random Forest} untuk prediksi PM\textsubscript{2,5} yang mencapai R² = 0,61, sedangkan \textcite{radjabaycolle2025improving} menerapkan LSTM dengan SHAP untuk prediksi kualitas udara Jakarta. Kedua penelitian Jakarta tersebut belum mengeksplorasi pendekatan \textit{hybrid} RF-ARIMA dengan \textit{feature engineering} sistematis.

\section{Kesenjangan Penelitian}

Berdasarkan kajian literatur, teridentifikasi beberapa kesenjangan penelitian yang menjadi peluang pengembangan.

Pertama, model \textit{hybrid} RF-ARIMA dari \textcite{yenkikar2025explainable} hanya menggunakan 6 fitur polutan mentah tanpa mengeksplorasi \textit{feature engineering}. Penelitian menunjukkan bahwa \textit{feature engineering} dapat meningkatkan performa prediksi secara signifikan \cite{naz2024two}, namun belum diterapkan pada arsitektur \textit{hybrid} RF-ARIMA.

Kedua, penelitian prediksi kualitas udara Jakarta yang ada belum mengeksplorasi pendekatan \textit{hybrid} RF-ARIMA. \textcite{alarsy2025pm25} hanya menggunakan \textit{Random Forest} tunggal, sedangkan \textcite{radjabaycolle2025improving} menggunakan LSTM tanpa komponen ARIMA untuk residual.

Ketiga, belum ada penelitian yang menerapkan model \textit{hybrid} RF-ARIMA dengan \textit{feature engineering} sistematis pada dataset Jakarta jangka panjang (lebih dari 10 tahun) untuk prediksi PM\textsubscript{10}.

Penelitian ini mengisi kesenjangan tersebut dengan mengembangkan model \textit{hybrid} RF-ARIMA dari \textcite{yenkikar2025explainable} melalui penambahan 10 fitur hasil \textit{feature engineering}, penerapan pada konteks Jakarta dengan data 15 tahun, fokus pada PM\textsubscript{10} yang relevan dengan sumber polusi Jakarta, dan integrasi SHAP untuk mengidentifikasi faktor kunci polusi yang mendukung kebijakan berbasis bukti.
