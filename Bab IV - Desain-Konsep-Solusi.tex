% ==========================================
% BAB IV DESAIN KONSEP SOLUSI
% ==========================================
\chapter{DESAIN KONSEP SOLUSI}
\label{chap:desain-konsep-solusi}

\section{Diagram Konseptual Sistem}

Bagian ini menjelaskan perbandingan sistem prediksi kualitas udara Jakarta sebelum dan sesudah pengembangan model \textit{hybrid} Random Forest-ARIMA dengan \textit{feature engineering}.

\subsection{Sistem Sebelum (\textit{Before})}

Sistem pemantauan kualitas udara Jakarta saat ini memiliki alur seperti ditunjukkan pada Gambar \ref{gambar:before}. Data ISPU yang terdiri dari lima polutan mentah (PM\textsubscript{10}, SO\textsubscript{2}, CO, O\textsubscript{3}, NO\textsubscript{2}) dikumpulkan dari stasiun pemantauan, kemudian dilakukan monitoring harian dan perhitungan nilai ISPU, lalu dilaporkan ke masyarakat sebagai status kualitas udara harian.

\begin{figure}[h]
  \centering
  \captionsetup{justification=centering}
      \includegraphics[width=0.5\textwidth]{image/gambar4.png}
  \caption{Diagram sistem pemantauan kualitas udara Jakarta saat ini}
  \label{gambar:before}
\end{figure}

Berdasarkan Gambar \ref{gambar:before}, sistem saat ini memiliki beberapa keterbatasan. Pertama, akurasi prediksi belum optimal karena tidak ada model \textit{machine learning} yang dioptimasi untuk prediksi PM\textsubscript{10}. Kedua, tidak ada \textit{feature engineering} sistematis karena hanya menggunakan lima polutan mentah tanpa mengeksplorasi pola temporal dan interaksi antar polutan. Ketiga, tidak ada \textit{explainability} untuk kebijakan karena sistem tidak memberikan informasi tentang faktor-faktor yang mendorong polusi.

\subsection{Sistem Sesudah (\textit{After})}

Sistem prediksi PM\textsubscript{10} Jakarta yang diusulkan memiliki alur seperti ditunjukkan pada Gambar \ref{gambar:after}. Sistem ini mengembangkan model \textit{hybrid} Random Forest-ARIMA dengan \textit{feature engineering} dan integrasi SHAP untuk \textit{explainability}.

\begin{figure}[h]
  \centering
  \captionsetup{justification=centering}
      \includegraphics[width=0.5\textwidth]{image/gambar5.png}
  \caption{Diagram sistem prediksi PM\textsubscript{10} Jakarta yang diusulkan}
  \label{gambar:after}
\end{figure}

Berdasarkan Gambar \ref{gambar:after}, sistem yang diusulkan memiliki enam tahap utama. Pertama, data ISPU Jakarta periode 2010--2025 dikumpulkan sebagai input. Kedua, dilakukan \textit{preprocessing} dan \textit{feature engineering} untuk menghasilkan 15 fitur dari 5 polutan mentah. Ketiga, data dibagi menjadi 80\% untuk pelatihan dan 20\% untuk pengujian (\textit{train-test split}). Keempat, model \textit{hybrid} Random Forest dan ARIMA dilatih menggunakan data pelatihan. Kelima, dilakukan analisis SHAP untuk mengidentifikasi kontribusi setiap fitur terhadap prediksi. Keenam, model dievaluasi menggunakan metrik RMSE, MAE, dan R². Output akhir berupa prediksi PM\textsubscript{10} harian beserta kontribusi fitur yang dapat digunakan untuk mendukung kebijakan.

\subsection{Perbandingan Sistem \textit{Before} dan \textit{After}}

Perbandingan antara sistem sebelum dan sesudah pengembangan ditunjukkan pada Tabel \ref{tbl:perbandingan}.

\begin{table}[h]
  \begin{tabular}{ | p{3.5cm} | p{4.5cm} | p{5cm} |}
  \hline
  \textbf{Aspek} & \textbf{Sistem Sebelum} & \textbf{Sistem Sesudah} \\
  \hline
  Data input & 5 polutan mentah & 15 fitur (5 baseline + 10 \textit{engineered}) \\
  \hline
  Model & Tidak ada model ML khusus & \textit{Hybrid} Random Forest-ARIMA \\
  \hline
  Akurasi target & Belum optimal & Target R² $\geq$ 0,85 \\
  \hline
  \textit{Explainability} & Tidak ada & Analisis SHAP (\textit{global} dan \textit{local}) \\
  \hline
  Pola temporal & Tidak dieksploitasi & Fitur lag, \textit{rolling}, dan temporal \\
  \hline
  Dukungan kebijakan & Tidak ada & Identifikasi faktor kunci polusi \\
  \hline
  \end{tabular}
\caption{Perbandingan sistem sebelum dan sesudah pengembangan}
\label{tbl:perbandingan}
\end{table}

\section{Penjelasan Desain Solusi}

Bagian ini menjelaskan desain setiap komponen sistem prediksi PM\textsubscript{10} Jakarta yang diusulkan secara ringkas dan relevan dengan masalah serta kebutuhan yang telah diidentifikasi pada Bab III.

\subsection{Desain \textit{Data Preprocessing}}

Tahap \textit{data preprocessing} bertujuan untuk menyiapkan data mentah agar dapat digunakan untuk pelatihan model. Desain \textit{preprocessing} mencakup dua proses utama.

Pertama, penanganan data hilang (\textit{missing value handling}) menggunakan metode \textit{rolling mean} dengan \textit{window} 5 hari. Metode ini dipilih karena menjaga kontinuitas pola temporal dan tidak menimbulkan kebocoran data dari masa depan (\textit{data leakage}). Berdasarkan analisis data, terdapat sekitar 5--6\% data hilang yang perlu ditangani.

Kedua, agregasi multi-stasiun dilakukan dengan menghitung rata-rata nilai polutan dari lima stasiun pemantauan (DKI1--DKI5) untuk setiap hari. Agregasi ini menghasilkan satu nilai representatif untuk Jakarta secara keseluruhan, menyederhanakan model tanpa kehilangan informasi penting.

Desain ini menjawab kebutuhan F6 (memproses data historis 2010--2025) dan NF2 (\textit{robustness}).

\subsection{Desain \textit{Feature Engineering}}

Tahap \textit{feature engineering} bertujuan untuk mengekstraksi informasi tambahan dari data mentah yang dapat meningkatkan akurasi prediksi. Desain \textit{feature engineering} menghasilkan 10 fitur tambahan dari 5 fitur baseline.

Fitur lag (pm10\_lag1, pm10\_lag2, pm10\_lag7) menangkap persistensi polusi dari hari-hari sebelumnya. Fitur \textit{rolling statistics} (pm10\_ma3, pm10\_ma7, pm10\_std7) menangkap tren jangka pendek dan volatilitas konsentrasi PM\textsubscript{10}. Fitur temporal (month, is\_weekend, season) menangkap pola musiman Jakarta, dengan musim kemarau (April--Oktober) cenderung memiliki polusi lebih tinggi dibandingkan musim hujan. Fitur interaksi (co\_times\_o3) menangkap aktivitas fotokimia yang berkontribusi pada pembentukan PM\textsubscript{10} sekunder.

Desain ini menjawab kebutuhan F3 (memanfaatkan minimal 15 fitur) dan didasarkan pada temuan \textcite{naz2024two} bahwa \textit{feature engineering} dapat meningkatkan performa model hingga 86\%.

\subsection{Desain Model \textit{Hybrid} Random Forest-ARIMA}

Tahap pemodelan menggunakan arsitektur \textit{hybrid} yang menggabungkan \textit{Random Forest} dan ARIMA secara sekuensial. Desain model terdiri dari empat langkah.

Pertama, data dibagi menjadi 80\% untuk pelatihan dan 20\% untuk pengujian menggunakan pembagian temporal (data terlama untuk pelatihan, data terbaru untuk pengujian). Kedua, \textit{Random Forest} dilatih menggunakan 15 fitur untuk memprediksi nilai PM\textsubscript{10}. Ketiga, residual (selisih antara nilai aktual dan prediksi \textit{Random Forest}) dihitung dan dimodelkan menggunakan ARIMA untuk menangkap pola temporal yang tidak tertangkap \textit{Random Forest}. Keempat, prediksi akhir dihitung sebagai penjumlahan prediksi \textit{Random Forest} dan prediksi residual ARIMA.

\textit{Hyperparameter tuning} akan dilakukan menggunakan \textit{Grid Search} dengan \textit{cross-validation} untuk memaksimalkan performa model. Parameter yang akan di-\textit{tuning} meliputi jumlah pohon (\textit{n\_estimators}), kedalaman maksimum (\textit{max\_depth}), dan \textit{order} ARIMA (p, d, q).

Desain ini menjawab kebutuhan F1 (prediksi PM\textsubscript{10} harian), F2 (model \textit{hybrid} RF-ARIMA), F4 (metrik performa), NF1 (akurasi 15\% lebih baik), dan NF2 (\textit{train-test split} 80:20).

\subsection{Desain \textit{Explainability} dengan SHAP}

Tahap \textit{explainability} menggunakan SHAP (\textit{SHapley Additive exPlanations}) untuk menjelaskan kontribusi setiap fitur terhadap prediksi model. Desain SHAP mencakup dua jenis interpretasi.

Interpretasi \textit{global} mengidentifikasi fitur-fitur yang secara umum paling berpengaruh terhadap prediksi PM\textsubscript{10} Jakarta. Visualisasi yang dihasilkan berupa \textit{summary plot} yang menunjukkan \textit{ranking} kontribusi fitur. Interpretasi \textit{local} menjelaskan faktor-faktor yang mendorong prediksi spesifik pada hari tertentu. Visualisasi yang dihasilkan berupa \textit{force plot} yang menunjukkan kontribusi positif dan negatif setiap fitur.

Informasi kontribusi fitur ini dapat digunakan oleh Dinas Lingkungan Hidup dan Dinas Kesehatan DKI Jakarta untuk mengidentifikasi sumber polusi utama dan merancang intervensi kebijakan yang tepat sasaran.

Desain ini menjawab kebutuhan F5 (kontribusi fitur menggunakan SHAP) dan NF3 (interpretabilitas \textit{global} dan \textit{local}).
